\pdfoutput=1

\documentclass{l4proj}

%
% put any packages here
%

\begin{document}
\title{Develop a computer algorithm for feature detection and classification from images using state-of-the-art deep learning methods}
\author{Ivelin Penchev}
\date{\today}
\maketitle

\begin{abstract}
Over the recent years feature detection in images and videos is becoming more and more popular. This report will try to explain the main principle behind how neural networks are used in such context. It will also discuss what the differences between several architectures of neural networks are and compare their results. Finally we shall use state of the art deep learning algorithms for feature recognition in images and incorporate them in real time video analysis.
\end{abstract}

\educationalconsent

\tableofcontents


\chapter{Introduction}
\pagenumbering{arabic}
\section{Motivation and context}

An Artificial Neural Network is learning model inspired by the biological structure of the brain. It is composed of a large number of neurons, just like the brain. The basic structure includes an input layer, several hidden layers and an output layer.
\Figure{h!}{12}{introduction/basic_structure}{The basic structure of a neural network}
Neural networks can be vastly different from one another - single direction logic with just several layers to multidirectional feedback network with more than ten layers. They are a powerful tool for solving many different problems, which would require difficult or very time consuming algorithms like digit or face recognition in images \cite{intro_book}. This chapter will take a closer look of some of the main characteristics of neural networks.


\section{What is a neural network}
As we discussed above a neural network consists of several layers of neurons but we haven't discussed what neurons actually are. The operation of a neuron consists of taking several inputs, weighting them and provide an output. There are many different types of neurons but the two which we will investigate here will be perceptrons and sigmoid neurons.

\subsection{Neurons} 
Neurons are the building blocks of a neural network. They receive one or more inputs and sums them to produce an output.

\Figure{h!}{5}{introduction/neuron}{A neuron}

The first type of neutor we will discuss is a perceptron. Perceptrons are functions that sums each input multiplied by a weight constant and if the sum is above a certain threshold they output one, otherwise output a zero \cite{intro_book}. The graph below shows the plot of the perceptron output, which is a very basic step function.

\begin{center}
    \begin{tikzpicture}
        \begin{axis}[
                xmin=-5,xmax=5,
                ymin=-0.1,ymax=1.1,
            ]
            \addplot[cmhplot,<-,domain=-5:0]{0};
            \addplot[cmhplot,->,domain=0:5]{1};
            \vasymptote {0};
        \end{axis}
    \end{tikzpicture}
\end{center}

\begin{eqnarray}
  \mbox{output} & = & \left\{ \begin{array}{ll}
      0 & \mbox{if } \sum_j w_j x_j \leq \mbox{ threshold} \\ 
      1 & \mbox{if } \sum_j w_j x_j > \mbox{ threshold}
      \end{array} \right.
\end{eqnarray}

The second type of neuron we will take a look at is sigmoid neurons. The basic operation of this type is still the same, the neuron receives inputs and computes an output. The difference is that rather than having binary inputs and output, the sigmoid neurons can take and produce any values between 0 and 1. In order to achieve this the output is defined as follow \cite{intro_book}:

\begin{center}
    \begin{tikzpicture}
        \begin{axis}[
                xmin=-5,xmax=5,
                ymin=-0.1,ymax=1.1,
            ]
        \addplot[cmhplot]{1/(1+e^(-x))};
        \end{axis}
    \end{tikzpicture}
\end{center}

\begin{eqnarray} 
  \sigma(z) \equiv \frac{1}{1+e^{-z}}.
\end{eqnarray}

The technical difference between the perceptron and sigmoid neuron is clearly the more inputs and outputs the sigmoid neuron can accept, but what does they mean for practical purposes? Let me illustrate that with an example. Lets assume that we have a neural network which needs to recognise a digit from an image. Lets assume that we will use 10 neurons in the output layer, each of which will represent one digit from 0 to 9. If we use perceptrons we can assume that if a certain perceptron outputs 1 the image is recognised as that digit. If we have multiple one outputs things get ambiguous and complicated quite quickly. Moreover, neural networks that use perceptrons are far harder to train, since small changes in weights may cause large changes in the output (more on that later).

On the other hand, sigmoid neurons are very flexible in this case. If we use sigmoid neurons we can choose the largest output of all 10 neurons and judge that that digit is in the image. Moreover, sigmoid neurons are much easier to train since small changes in one neuron cause small changes in the output of the network.

All in all, both perceptrons and sigmoid neurons are used in neural networks and are good in certain areas. The focus of this study will fall on the sigmoid neurons since they are more powerfull and suit our needs better.

\subsection{Architecture}
//Comment on single hidden layer networks and more complicated architectures like convolutional networks and techniques used there.

\section{Project description}
//Unrelated paragraph I wrote and didn't want to delete. Keep here for now

The cycle of developing a neural network consists of several steps: defining the architecture of the neural network - number of inputs, number of hidden layers and number of neurons in them, etc. Afterwards the network needs to be trained, which is achieved by supplying many test cases through the network and verifying the output of the network and modifying it accordingly. At first the behaviour would be random but soon after the network starts learning and optimising itself which increases the fitness of the network.

\bibliographystyle{plain}
\bibliography{bib}

\end{document}